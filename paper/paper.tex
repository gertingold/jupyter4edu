%%%%%%%%%%%%%%%%%%%%%%% file template.tex %%%%%%%%%%%%%%%%%%%%%%%%%
%
% This is a general template file for the LaTeX package SVJour3
% for Springer journals.          Springer Heidelberg 2010/09/16
%
% Copy it to a new file with a new name and use it as the basis
% for your article. Delete % signs as needed.
%
% This template includes a few options for different layouts and
% content for various journals. Please consult a previous issue of
% your journal as needed.
%
%%%%%%%%%%%%%%%%%%%%%%%%%%%%%%%%%%%%%%%%%%%%%%%%%%%%%%%%%%%%%%%%%%%
%
% First comes an example EPS file -- just ignore it and
% proceed on the \documentclass line
% your LaTeX will extract the file if required
\RequirePackage{fix-cm}
%
%\documentclass{svjour3}                     % onecolumn (standard format)
%\documentclass[smallcondensed]{svjour3}     % onecolumn (ditto)
%\documentclass[smallextended]{svjour3}       % onecolumn (second format)
\documentclass[twocolumn]{svjour3}          % twocolumn
%
\smartqed  % flush right qed marks, e.g. at end of proof
%
\usepackage{graphicx}
%
% \usepackage{mathptmx}      % use Times fonts if available on your TeX system
%
% insert here the call for the packages your document requires
\usepackage{url}
%\usepackage{latexsym}
% etc.
%
% please place your own definitions here and don't use \def but
% \newcommand{}{}
%
% Insert the name of "your journal" with
\journalname{Journal of Science Education and Technology}
%
\begin{document}

\title{Experiences in teaching with Jupyter with nbgrader}
%about the article that should go on the front page should be
%placed here. General acknowledgments should be placed at the end of the article.}
% \subtitle{Do you have a subtitle?\\ If so, write it here}

%\titlerunning{Short form of title}        % if too long for running head

\author{Marcin Kostur \and
        Marina Appiou-Nikiforou \and
        Andreas Papadopoulos \and
	Gert-Ludwig Ingold
}

%\authorrunning{Short form of author list} % if too long for running head

\institute{Marcin Kostur \at
              first address \\
              Tel.: +123-45-678910\\
              Fax: +123-45-678910\\
              \email{marcin.kostur@us.edu.pl}           %  \\
%             \emph{Present address:} of F. Author  %  if needed
           \and
           Marina Appiou-Nikiforou \at
              European University Cyprus, 
	      6 Diogenous Street, 1516 Nicosia, Cyprus\\
              \email{M.Nikiforou@euc.ac.cy}
	   \and
	   Andreas Papadopoulos \at
              European University Cyprus, 
	      6 Diogenous Street, 1516 Nicosia, Cyprus \\
              \email{apapado89@gmail.com}
	     % \emph{Present address:}
	   \and
	   Gert-Ludwig Ingold \at
	      Institut f{\"u}r Physik, Universit{\"a}t Augsburg,
	      Universit{\"a}tsstra{\ss}e 1, 86135 Augsburg, Germany\\
	      \email{gert.ingold@physik.uni-augsburg.de}
}

\date{Received: date / Accepted: date}
% The correct dates will be entered by the editor


\maketitle

\begin{abstract}
Insert your abstract here. Include keywords, PACS and mathematical
subject classification numbers as needed.
\keywords{First keyword \and Second keyword \and More}
% \PACS{PACS code1 \and PACS code2 \and more}
% \subclass{MSC code1 \and MSC code2 \and more}
\end{abstract}

\section{Introduction}
\label{intro}

Your text comes here. Separate text sections with

\section{Section title}
\label{sec:1}
We discuss Jupyter \cite{jupyter-edu-book} and nbgrader \cite{nbgrader}.

\subsection{Subsection title}
\label{sec:2}

%% For one-column wide figures use
%\begin{figure}
%% Use the relevant command to insert your figure file.
%% For example, with the graphicx package use
%  \includegraphics{example.eps}
%% figure caption is below the figure
%\caption{Please write your figure caption here}
%\label{fig:1}       % Give a unique label
%\end{figure}
%%
%% For two-column wide figures use
%\begin{figure*}
%% Use the relevant command to insert your figure file.
%% For example, with the graphicx package use
%  \includegraphics[width=0.75\textwidth]{example.eps}
%% figure caption is below the figure
%\caption{Please write your figure caption here}
%\label{fig:2}       % Give a unique label
%\end{figure*}
%%
%% For tables use
%\begin{table}
%% table caption is above the table
%\caption{Please write your table caption here}
%\label{tab:1}       % Give a unique label
%% For LaTeX tables use
%\begin{tabular}{lll}
%\hline\noalign{\smallskip}
%first & second & third  \\
%\noalign{\smallskip}\hline\noalign{\smallskip}
%number & number & number \\
%number & number & number \\
%\noalign{\smallskip}\hline
%\end{tabular}
%\end{table}


\begin{acknowledgements}
This work was supported through the Erasmus+ programme of the
European Union.
\end{acknowledgements}


\section*{Conflict of interest}
The authors declare that they have no conflict of interest.


% BibTeX users please use one of
%\bibliographystyle{spbasic}      % basic style, author-year citations
%\bibliographystyle{spmpsci}      % mathematics and physical sciences
%\bibliographystyle{spphys}       % APS-like style for physics
%\bibliography{}   % name your BibTeX data base

% Non-BibTeX users please use
\begin{thebibliography}{99}
%
% and use \bibitem to create references. Consult the Instructions
% for authors for reference list style.
%
\bibitem{jupyter-edu-book}
Barba, L. A., Barker, L. J., Blank, D. S., Brown, J, Downey, A. B., George, T.,
Heagy, L. J., Mandli, K. T., Moore, J. K., Lippert, D., Niemeyer, K. E.,
Watkins, R. R., West, R. H., Wickes, E., Willing, C., \& Zingale M. (2019).
Teaching and Learning with Jupyter,
\url{https://jupyter4edu.github.io/jupyter-edu-book/}.
\bibitem{nbgrader}
% https://doi.org/10.21105/jose.00032 
Project Jupyter, Blank, D., Bourgin, D., Brown, A., Bussonnier, M.,
Frederic, J., Granger, B., Griffiths, T. L.,  Hamrick, J., Kelley, K.,
Pacer, M., Page, L., P{\'e}rez, F., Ragan-Kelley, B., Suchow, J. W.,
\& Willing, C. (2019).
nbgrader: A Tool for Creating and Grading Assignments in the Jupyter Notebook,
\textit{Journal of Open Source Education}, 2(11), 32.
% Format for books
%\bibitem{RefB}
%Author, Book title, page numbers. Publisher, place (year)
% etc
\end{thebibliography}

\end{document}
% end of file template.tex

